\section{Carta Gantt con línea base y desviaciones}
Incluir cada Gantt con la explicación del cambio y el efecto en la planificación global
\section{Riesgos de Alto nivel (Amenazas), Impacto, estrategia}
Explique los Riesgos que tengan impacto en su proyecto. Ordene los riesgos y defina las acciones (estrategia)que se proponen para abordarles.
Incluir columna con los riesgos que se presentaron en el proyecto.
        
\section{Estimación CU}
Estimación de tamaño de Sw: Puntos de Casos de Uso
    • Clasificar Actores
    • Clasificar casos de uso
    • Factores técnicos
    • Factores del entorno
    • Calcular puntos de Casos de uso


Tipo de caso de uso 
5   Simple      Menos de 5 clases 5    3 transacciones o menos
10  Medio       5 a 10 clases 10       4 a 7 transacciones
15  Complejo    Más de 10 clases 18    Más de 7 transacciones

Tipo de actor D

1   Simple  Otro sistema que interactúa con el sistema a desarrollar mediante una interfaz de programación (API).
2   Medio   Otro sistema interactuando a través de un protocolo (ej. TCP/IP) o una persona interactuando a través de una interfaz en modo texto
3   Complejo    Una persona que interactúa con el sistema mediante una interfaz gráfica (GUI).

\begin{itemize}
    \item Calcular UUCP (Unadjusted Use Case Point)
    \item UUCP= UAW+UUCW 
    \item Calcular  TCF (Technical Complexity Factor)
    \item TCF=0.6+(0.01*TFactor)
    \item Calcular  EF (Environmental Factor)
    \item EF=1.4+(-0.03*EFactor)
    \item UCP = UUCP * TCF * EF
\end{itemize}

Evaluación de relevancia de factores técnicos y ambientales     Valor
Irrelevante De                                                  0 a 2. 
Medio De                                                        3 a 4. 
Esencial                                                        5


Calculate TCF (Technical Complexity Factor)

Technical Factor    Multiplier      Relevancia percibida    Resultado multiplicación
Distributed System  2
Application performance objectives, in either response or throughput    1
End-user efficiency (on-line)   1
Complex internal processing     1
Reusability, the code must be able to reuse in other applications   1
Installation ease   0,5
Operational ease, usability     0,5
Portability     2
Changeability   1
Concurrency     1
Special security features   1
Provide direct access for third parties 1
Special user training facilities    1



Environmental Factor        Multiplier      Relevancia percibida    Resultado multiplicación
Familiar with Objectory + RUP   1,5         
Application experience          0,5
Object Oriented experience      1
Analyst capability              0,5
Motivation                      1
Stable requirements             2
Par time workers                -1
Difficult programming language  -1


Level of Effort. Schneider and Winters, proponen que: Si la suma entre (el número de factores de entorno (F1 a F6) inferiores a 3 y el número de factores de entorno (F7 a F8) superiores a 3). 
\begin{itemize}
    \item  es menor o igual a 2 entonces LOE=20, 
    \item  es 3 o 4 LOE=28. 
    \item  es mayor a 4 reconsiderar el proyecto. Por ejemplo, reducir los riesgos relacionados con los factores de entorno.
\end{itemize}


\section{Resumen Esfuerzo}

El final de este documento se debe indicar las horas destinadas en realizar cada una de las fases del desarrollo del software, las horas corresponden a la suma de las horas gastadas por cada integrante y del equipo en conjunto.

Tabla

Actividades/fases/casos de Uso          N° Horas
Cuantas horas se dedicaron en)          
Cuantas horas se dedicaron en           
Cuantas horas se dedicaron en           
Cuantas horas se dedicaron en programar 
Cuantas horas se dedicaron en informe completo (preparar y corregir)    
Cuantas horas se dedicaron a git        
TOTAL                           