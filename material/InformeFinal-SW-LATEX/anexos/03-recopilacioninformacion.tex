<Todas las técnicas aplicadas y el respaldo de la información recopilada. Por ejemplo, entrevista, cuestionarios, observación en terreno, revisión de documentación, talleres grupales, etc.
a.	En cuestionarios se indica: a quien, para que, cuando se aplicó, y las preguntas. Después se incluyen las tablas de respuestas y resumen de los resultados.
b.	En entrevistas se indica: a quien, para que, cuando se aplicó, y preguntas. Después se incluyen las respuestas y firma del cliente.
c.	Observación en terreno: donde se realizó, que proceso observó, que usuarios, que información recopiló.
d.	Revisión de documentos (internos o externos): que documentos obtuvo, desde donde, cuando los obtuvo, que información útil extrajo.
Toda la información que se recopila de las técnicas debe estar relacionada con el contenido del informe, es decir con la etapa del desarrollo del software. Por ejemplo, si no estamos interesados de los tipos de usuarios del software, no debe preguntar o extraer información al respecto.>