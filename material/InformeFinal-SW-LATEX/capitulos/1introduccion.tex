Introducción al tema




\section{Descripción del Problema}

Describir el problema


\subsection{Descripción de la Organización y Área de trabajo}

En el texto responda a las preguntas: para que empresa/institución, quienes son, donde se ubican, desde cuándo, que hacen u ofrecen, quienes son sus clientes. 
En qué ámbito de la empresa/institución se enmarca este proyecto.

\subsection{Proceso de Negocio Actual}

Explique la situación actual a través de una descripción de los procesos o actividades que han dado origen a este proyecto. 

\subsection{Explicación del proceso de negocio}

Utilice cualquier tipo de diagrama, por ejemplo, 
diagrama de procesos de negocios (notación BPMN), o 
diagrama de actividad (UML 2.0) o 
diagrama de procedimiento adm.
 
Recuerde referenciar la figura en un texto como por EJEMPLO 

“A continuación, la Figura 1 representa los procedimientos que se siguen actualmente en la empresa.” Debe explicar el modelo, tal como “leer el modelo” sin entrar en detalles innecesarios. 
Incluya título para la figura, por ejemplo “Figura 1: Diagrama de Actividad, notación UML2.0 para representar el procedimiento para …..”.

\section{Definición de usuarios}
\subsection{Caracterización de los usuarios}
\subsection{Problemas de información de los usuarios}

\subsection{Oportunidades de Mejora o Problemáticas}

Se identifica y especifica el problema o la oportunidad de mejora que ha motivado la necesidad del sistema, lo cual definirá el objetivo del sistema. NO SE COMENTA LA SOLUCIÓN.

\subsection{Propuesta de solución}

Debe explicar en términos generales cómo las TIC pueden resolver o mejorar la(s) problemática identificada y quienes serán los usuarios principales, que tecnología se utilizaría para dar soporte a la propuesta. 

\section{Soluciones Similares disponibles}

Se investigó en biblioteca Werken, en el buscador de Google, Play Store y en App Store con fecha xxxxxxxxxxxxxx con el propósito de conocer que otras soluciones existen actualmente. 


\section{Justificación del Proyecto}

Comente las razones técnicas, económicas, funcionales, sociales etc por las que ESTE PROYECTO ES importante que sea desarrollado.
Esto generalmente se escribe inicialmente en la propuesta del tema.


\section{Objetivos del proyecto}
\subsection{Objetivo general}

Objetivos generales del proyecto, estos objetivos son distintos a los objetivos del software/sistema de Sw. Utilice sólo1 verbo activo por objetivo
Los Objetivos del proyecto terminan con el proyecto y los objetivos del software se logran con el uso del software, es decir van más allá de la fecha de término del proyecto. 
Por ejemplo, un objetivo del proyecto puede comenzar como “implementar una solución a…”


\subsection{Objetivos específicos}

Utilice sólo1 verbo activo por objetivo, No confunda los objetivos con las actividades que serán desarrolladas, la lógica o relación es que los objetivos son metas, se pueden ser alcanzables una vez terminadas varias actividades.
hacer entrevistas, cuestionarios
Revisar información 
Evaluar procesos
proponer la solución Sw obj

\subsection{Actividades para Realización del Proyecto}

En la siguiente sección se describe la actividades a realizar para la investigación, por objetivos específicos.
\begin{itemize}
    \item 
    \item 
    \item 
    \item 
    \item 
    \item 
    
\end{itemize}



\section{Composición del Informe} 

El presente trabajo se encuentra dividido en xx capítulos. A continuación se describe brevemente el contenido de cada uno de ellos.

no usar viñetas, solo párrafos....\\


