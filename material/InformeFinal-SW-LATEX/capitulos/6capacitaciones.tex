
La capacitación de contenidos y conceptos relacionados al sw, entrenamiento en el uso del software, en la resolución de problemas, por ejemplo. Considerando a los distintos usuarios y su nivel de expertiz. Estas actividades pueden ser online o presencial y requieren que el Sw se encuentre disponible.
Implantación considera el proceso en el que la empresa adopta el sw y la nueva forma de hacer las cosas, existen estrategias revise cuál de ellas y justifique por que será utilizada. Por ejemplo, radical/directa, paralelo, entre otras. Los tipos de implantación son:
\begin{itemize}
    \item Sistemas paralelos: es el método más seguro, el cual consiste en poner a trabajar los dos sistemas en paralelo, de esta manera los usuarios siguen utilizando el sistema anterior de manera acostumbrada, aunque van teniendo más contacto con el otro. La data va a ser poco a poco migrada de un sistema a otro y sin que el usuario se dé cuenta vamos obligándolo a usar poco a poco más el nuevo sistema. Una de las desventajas es que al estar operando los dos sistemas los costos se duplicaran debido a que pudiera ser que se tenga que contratar personal para que opere los dos sistemas, puede que también el nuevo sistema sea rechazado por los usuarios y se vuelva al sistema anterior.
    \item Conversión directa: este tipo de conversión se hace de manera radical debido que se hace de un día a otro obligando tanto físico como psicológicamente al usuario que no existe otro sistema y debe usar ese. Esto tiene una desventaja ya que al eliminar por completo el sistema antiguo se quedan sin respaldo, y si el sistema nuevo llegase a tener problemas este quedara parando a la empresa hasta que se solucione, también la empresa se retrasa varias semanas debido que toda la captura de datos debe empezarse de nuevo y los departamentos deben ponerse a trabajar con eso. una vez que empiece este proceso debe seguirse a pesar de las frustraciones que puede haber por cuestión de tiempo perdido. Este método necesita una buena planificación, para que así no exista perdida de ningún tipo.
    \item Enfoque piloto:   este método funciona de la siguiente manera, tenemos el sistema, pero solo se lo aplicamos a un departamento a manera de prueba para así también ir probándolo y mejorándolo una vez capaces de trabajar con él, y saber que el sistema está trabajando en su plenitud y no tiene errores y ha minimizado tareas en ese departamento tanto como costos, tiempo etc. se va a implementar en toda la empresa.
    \item Modelo por etapas: este método se da debido a la tardanza de la llegada del nuevo sistema que pasara de días a meses y es por eso que solo algunos tendrán acceso a él. Ejemplo: soy un empresario, tengo 15 tiendas de ropa, automatizar a las 15 tiendas es muy costoso y es por eso que la implanto
\end{itemize} primero en 5 tiendas y luego en el resto.
.
Preparación de datos / Migración /Poblamiento debe calendarizar esta etapa y documentar el proceso en caso que deba ser repetido.
Puesta en marcha planificar tiempo de monitoreo y la forma como se atenderán las consultas del usuario hasta que finalmente sea liberado el sw.
Todos estos elementos deben ser definido y justificado y luego calendarizado en una Gantt.


\section{Estado del Proyecto}

En que etapa se encuentra, que falta ¿??