  \documentclass[final,letterpaper,oneside,authoryear,11pt,singlespace,spanish]{ezthesis}
\usepackage[spanish]{babel}
\addto\captionsspanish{
\def\bibname{Referencias}
\def\tablename{Tabla}
}
\usepackage{hyperref}
\hypersetup{
    colorlinks=true,
    linkcolor=blue,
    filecolor=blue,      
    urlcolor=blue,
    citecolor=blue
}
\usepackage{rotating}
%%\usepackage{algorithm}
%\usepackage{algpseudocode}
%\usepackage[noend]{algpseudocode}
\usepackage{siunitx}
\usepackage{mathtools}
\usepackage[T1]{fontenc}
\usepackage[utf8]{inputenc}
\usepackage{anysize}
\usepackage{txfonts}
\usepackage{wrapfig} %figura entre texto
\usepackage{times}
\usepackage{listings}
\usepackage{acronym}
\usepackage{array}
\usepackage{textcomp}
\usepackage{fancyvrb}
\usepackage{fancyhdr}
\usepackage{wallpaper}
\usepackage{verbatim}%Agregado por Luis
\usepackage{latexsym}
\usepackage{url}
\usepackage{multicol}
\usepackage{graphicx}
\usepackage{multirow}
\usepackage{float}
\usepackage{lmodern}
\usepackage{eso-pic}
\usepackage{subfig}
%\usepackage{amssymb}
\usepackage{color}
\usepackage{ragged2e}
\usepackage[all]{xy}
\usepackage{xspace,epic,eepic}
\usepackage{algorithmic}
\usepackage[ruled,vlined]{algorithm2e}
%\usepackage[ruled,vlined,linesnumbered,titlenotnumbered, %portuguese]{algorithm2e}
\usepackage{enumitem}
\setlist{topsep=0pt,noitemsep}
\setcounter{tocdepth}{3}
\marginsize{3cm}{2.5cm}{2.5cm}{2.5cm}

%Definicion de Colores
\definecolor{gray97}{gray}{.97}
\definecolor{gray75}{gray}{.75}
\definecolor{gray45}{gray}{.45}
\definecolor{listinggray}{gray}{0.9}
\definecolor{lbcolor}{rgb}{0.9,0.9,0.9}
\newcommand\rojo[1]{\textcolor[rgb]{1,0,0}{\textbf{#1}}}
\newcommand\red[1]{\textcolor[rgb]{1.00,0.00,0.00}{#1}}
\newcommand\azul[1]{\textcolor[rgb]{0,0,1}{\textbf{#1}}}
\newcommand\blue[1]{\textcolor[rgb]{0,0,1}{{#1}}}
\newcommand\verde[1]{\textcolor[rgb]{0,.5,0.2}{\textbf{#1}}}
\newcommand\naranjo[1]{\textcolor[rgb]{1.00,0.36,0.06}{\textbf{#1}}}
%\ifCLASSINFOpdf
%\else
%\fi
%\hyphenation{pa-la-bra}

\lstset{
	%backgroundcolor=\color{lbcolor},
	tabsize=4,
	rulecolor=,
	language=SQL,
        basicstyle=\small,
        upquote=true,
        aboveskip={1.5\baselineskip},
        columns=fixed,
        showstringspaces=false,
        extendedchars=true,
        breaklines=true,
        prebreak = \raisebox{0ex}[0ex][0ex]{\ensuremath{\hookleftarrow}},
        %frame=single,
        showtabs=false,
        showspaces=false,
        showstringspaces=false,
        identifierstyle=\ttfamily,
        keywordstyle=\color[rgb]{0,0,0},
        commentstyle=\color[rgb]{0,0,0},
        stringstyle=\color[rgb]{0,0,0},
}


%\renewcommand{\labelenumi}{\arabic{enumi}.} % (1., 2., 3.,...)
%\renewcommand{\labelenumi}{\roman{enumi}.} %  (i., ii., iii.,...)
%\renewcommand{\labelenumi}{\Roman{enumi}.} %  (I., II., III.,...)
%\renewcommand{\labelenumi}{\alph{enumi}.}   % (a., b., c.,...)
\renewcommand{\labelenumi}{(\alph{enumi})} % [(a), (b), (c),...]
%\renewcommand{\labelenumi}{\Alph{enumi}.}  %  (A., B., C.,...)


\newtheorem{theorem}{Theorem}[chapter]
\newtheorem{teorema}[theorem]{Teorema}
\newtheorem{lem}{Theorem}[chapter]
\newtheorem{lema}[lem]{Lema}
\newtheorem{prop}{Theorem}[chapter]
\newtheorem{proposition}[prop]{Proposición}
\newtheorem{coro}{Theorem}[chapter]
\newtheorem{corolario}[coro]{Corolario}
\newtheorem{exam}{Theorem}[chapter]
\newtheorem{example}[exam]{Ejemplo}
\newtheorem{test}{Theorem}[chapter]
\newtheorem{prueba}[test]{Prueba}
\newtheorem{defi}{Theorem}[chapter]
\newtheorem{definition}[defi]{Definición}
\newtheorem{obs}{Theorem}[chapter]
\newtheorem{observacion}[obs]{Observación}

\newcommand{\keywords}[1]{\par\addvspace\baselineskip
\noindent\keywordname\enspace\ignorespaces#1}
\renewcommand{\abstractname}{\prefacesection{Resumen}}
\renewcommand{\listtablename}{Índice de Tablas}
\renewcommand{\tablename}{Tabla}
\renewcommand{\refname}{Bibliografía}
\newcommand{\boxtheorem}{\hfill $\Box$}
%%%%%%%%IEEE Palabras Claves
%\renewcommand{\IEEEkeywords}{\textbf{\emph{Palabras Clave---}}}


\newcommand{\qed}{\nobreak \ifvmode \relax \else
      \ifdim\lastskip<1.5em \hskip-\lastskip
      \hskip1.5em plus0em minus0.5em \fi \nobreak
      \vrule height0.75em width0.5em depth0.25em\fi}

\author{XXXX XXXX XXXXX}
\title{XXXXX}
\degree{Ingeniería Civil Informática}
\supervisor{XXXX XXXXX XXXXX }
\institution{Universidad del Bío-Bío, Chile}
\faculty{Facultad de Ciencias Empresariales}
\department{Departamento de Sistemas de Información}


\begin{document}
\hyphenation{com-pu-ta-dor}
\cleardoublepage
\pagenumbering{roman}
\setcounter{page}{1}
%% En esta secci'on se describe la estructura del documento de la tesis.
%% Consulta los reglamentos de tu universidad para determinar el orden
%% y la cantidad de secciones que debes de incluir

%% # Portada de la tesis #
%% Mirar el archivo "titlepage.tex" para los detalles.

%% ## Construye tu propia portada ##
%%
%% Una portada se conforma por una secuencia de "Blocks" que incluyen
%% piezas individuales de informaci'on. Un "Block" puede incluir, por
%% ejemplo, el t'itulo del documento, una im'agen (logotipo de la universidad),
%% el nombre del autor, nombre del supervisor, u cualquier otra pieza de
%% informaci'on.
%%
%% Cada "Block" aparece centrado horizontalmente en la p'agina y,
%% verticalmente, todos los "Blocks" se distruyen de manera uniforme
%% a lo largo de p'agina.
%%
%% Nota tambi'en que, dentro de un mismo "Block" se pueden cortar
%% lineas usando el comando \\
%%
%% El tama'no del texto dentro de un "Block" se puede modificar usando uno de
%% los comandos:
%%   \small      \LARGE
%%   \large      \huge
%%   \Large      \Huge
%%
%% Y el tipo de letra se puede modificar usando:
%%   \bfseries - negritas
%%   \itshape  - it'alicas
%%   \scshape  - small caps
%%   \slshape  - slanted
%%   \sffamily - sans serif
%%
%% Para producir plantillas generales, la informaci'on que ha sido inclu'ida
%% en el archivo principal "tesis.tex" se puede accesar aqu'i usando:
%%   \insertauthor
%%   \inserttitle
%%   \insertsupervisor
%%   \insertinstitution
%%   \insertdegree
%%   \insertfaculty
%%   \insertdepartment
%%   \insertsubmitdate
\begin{titlepage}
  \TitleBlock{\includegraphics[height=3cm]{figures/UBB.png}}
  \TitleBlock{\scshape\insertinstitution}
  \TitleBlock[\bigskip]{\scshape\insertfaculty}
  \TitleBlock[\bigskip]{\insertdepartment}
  \TitleBlock{\Huge\scshape\inserttitle}
  \TitleBlock{\scshape
    Anteproyecto de título presentado por\\
     \insertauthor \\
    de la Carrera \insertdegree\\
    Dirigida por \insertsupervisor}
  \TitleBlock{\insertsubmitdate}

%Proyecto de Título para optar al título de Ingeniero de Ejecución en Computación e Informática 

%Proyecto de Título para optar al título de Ingeniero Civil en Informática 


\end{titlepage}



%% # Prefacios #
%% Por cada prefacio (p.e. agradecimientos, resumen, etc.) crear
%% un nuevo archivo e incluirlo aqu'i
%% Para m'as detalles y un ejemplo mirar el archivo "gracias.tex".

%
\chapter*{Resumen\markboth{Resumen}{Resumen}}

\renewcommand{\keywords}{\textbf{\emph{Palabras Clave ---~}}}

Debe describir su proyecto, resultados y beneficios esperados.



\keywords{xxxx,xxxxx,xxxxx} 


\chapter*{Resumen\markboth{Resumen}{Resumen}}

\renewcommand{\keywords}{\textbf{\emph{Palabras Clave ---~}}}

Debe describir su proyecto, resultados y beneficios esperados.



\keywords{xxxx,xxxxx,xxxxx} 


\chapter*{Dedicatoria y/o Agradecimientos\markboth{Dedicatoria}{Dedicatoria}}


Debe describir su proyecto, resultados y beneficios esperados.






\chapter*{Acronimos\markboth{Acronimos}{Acronimos}}






\tableofcontents
%\listoffigures
\renewcommand{\listfigurename}{Índice de Figuras}
\listoffigures
\renewcommand{\listtablename}{Índice de Tablas}
\listoftables

\cleardoublepage
\pagenumbering{arabic}
\setcounter{page}{1}

\acrodef{VB}{Visula Basic}
\acrodef{FP}{Formación Profesional}
\acrodef{SIDA}{Síndrome de Inmunodeficiencia Adquirida}

\chapter{Introducción}  \label{cap:introduccion}
Introducción al tema




\section{Descripción del Problema}

Describir el problema


\subsection{Descripción de la Organización y Área de trabajo}

En el texto responda a las preguntas: para que empresa/institución, quienes son, donde se ubican, desde cuándo, que hacen u ofrecen, quienes son sus clientes. 
En qué ámbito de la empresa/institución se enmarca este proyecto.

\subsection{Proceso de Negocio Actual}

Explique la situación actual a través de una descripción de los procesos o actividades que han dado origen a este proyecto. 

\subsection{Explicación del proceso de negocio}

Utilice cualquier tipo de diagrama, por ejemplo, 
diagrama de procesos de negocios (notación BPMN), o 
diagrama de actividad (UML 2.0) o 
diagrama de procedimiento adm.
 
Recuerde referenciar la figura en un texto como por EJEMPLO 

“A continuación, la Figura 1 representa los procedimientos que se siguen actualmente en la empresa.” Debe explicar el modelo, tal como “leer el modelo” sin entrar en detalles innecesarios. 
Incluya título para la figura, por ejemplo “Figura 1: Diagrama de Actividad, notación UML2.0 para representar el procedimiento para …..”.

\section{Definición de usuarios}
\subsection{Caracterización de los usuarios}
\subsection{Problemas de información de los usuarios}

\subsection{Oportunidades de Mejora o Problemáticas}

Se identifica y especifica el problema o la oportunidad de mejora que ha motivado la necesidad del sistema, lo cual definirá el objetivo del sistema. NO SE COMENTA LA SOLUCIÓN.

\subsection{Propuesta de solución}

Debe explicar en términos generales cómo las TIC pueden resolver o mejorar la(s) problemática identificada y quienes serán los usuarios principales, que tecnología se utilizaría para dar soporte a la propuesta. 

\section{Soluciones Similares disponibles}

Se investigó en biblioteca Werken, en el buscador de Google, Play Store y en App Store con fecha xxxxxxxxxxxxxx con el propósito de conocer que otras soluciones existen actualmente. 


\section{Justificación del Proyecto}

Comente las razones técnicas, económicas, funcionales, sociales etc por las que ESTE PROYECTO ES importante que sea desarrollado.
Esto generalmente se escribe inicialmente en la propuesta del tema.


\section{Objetivos del proyecto}
\subsection{Objetivo general}

Objetivos generales del proyecto, estos objetivos son distintos a los objetivos del software/sistema de Sw. Utilice sólo1 verbo activo por objetivo
Los Objetivos del proyecto terminan con el proyecto y los objetivos del software se logran con el uso del software, es decir van más allá de la fecha de término del proyecto. 
Por ejemplo, un objetivo del proyecto puede comenzar como “implementar una solución a…”


\subsection{Objetivos específicos}

Utilice sólo1 verbo activo por objetivo, No confunda los objetivos con las actividades que serán desarrolladas, la lógica o relación es que los objetivos son metas, se pueden ser alcanzables una vez terminadas varias actividades.
hacer entrevistas, cuestionarios
Revisar información 
Evaluar procesos
proponer la solución Sw obj

\subsection{Actividades para Realización del Proyecto}

En la siguiente sección se describe la actividades a realizar para la investigación, por objetivos específicos.
\begin{itemize}
    \item 
    \item 
    \item 
    \item 
    \item 
    \item 
    
\end{itemize}



\section{Composición del Informe} 

El presente trabajo se encuentra dividido en xx capítulos. A continuación se describe brevemente el contenido de cada uno de ellos.

no usar viñetas, solo párrafos....\\




\chapter{Marco Teórico} \label{cap:marco_teorico}

El presente capítulo tiene por objetivo describir los principales conceptos asociados a la formación de complejos proteicos y a su predicción, facilitando al lector la comprensión de las secciones posteriores.

\section{Temas que se abordarán en el proyecto: IA}

\subsection{del area a trabajar dentro de IA}
\subsection{del area a trabajar dentro de IA}
\begin{enumerate}
  \item \textbf{Clasificación de las interfase PPI}
  \item \textbf{Fuerzas que intervienen:} 
\end{enumerate}


\section{otro tema : ACP}\label{sec:docking_proteina_proteina}


A continuación se describen de forma más detallada las características más importantes de los procesos de búsqueda y puntuación.




\chapter{Estado del Arte} 
Introducción al capítulo

\section{Planificación de la revisión de la literatura}

\section{Resultados de la revisión de la literatura}


\subsection{Trabajos realizados en general en el área:  IA}
\subsection{Trabajos enfocados en el área que se esta trabajando: Machine learning}
\subsection{Trabajos enfocados especificamente en lo que se esta realizando, pero existen diferencias: Machine learnig en imagenes 2D}


\label{cap:estado_del_arte}

\chapter{Desarrollo del Trabajo} \label{cap:metodologia}

En el presente capítulo se explica el desafío abordado y la solución propuesta en este trabajo. Luego, se describen los algoritmos y programas utilizados como base para obtener la puntuación y ranking de las poses de docking proteína-proteína obtenidas por los usuarios de la aplicación. Por último, se detalla el desarrollo del sistema propuesto.

\section{Planificación de la experimentación}

\section{Problema a abordar}
\section{Solución propuesta}
\section{Algoritmo utilizado: Método Formas de Contexto}
\subsubsection{Filtros aplicados}
\subsection{Capas superficiales}
\subsection{Viabilidad de una pose}
\subsection{Puntuación de la pose $\pi$}
Si la pose evaluada no presenta superposición grande o aguda, es puntuada empleando para ello el cálculo del BSA. Luego, se clasifica junto al resto de formas de contexto.

\subsection{Datos de Entrada}
\section{Softwares y lenguajes utilizados}
\subsection{Plataforma de Desarrollo}
\subsection{UCSF Chimera}
\subsection{Programa MSMS}
\subsection{Blender}
\subsection{Software VMD}
\section{Lenguajes}
\section{Metodología}
\subsection{Procedimiento para el cálculo de formas de contexto}
\subsection{Procedimiento en Unity}


\textcolor{red}{La sección de pruebas y validaciones van en este capitulo}

\chapter{Resultados}  \label{cap:aplicacion}
En este capítulo se presenta el software desarrollado luego de llevar a cabo la metodología descrita en la sección \ref{cap:metodologia} y se realiza una breve discusión sobre lo obtenido. Como se mencionó previamente, para crear el sistema se utilizó el motor de videojuegos Unity 3D.
\section{Aplicación Desarrollada - si es que la hay}
\section{Discusión}


\chapter{Conclusiones} \label{cap:conclusion}
A partir de estructuras individuales de proteínas contenidas en archivos PDB

\subsection{Trabajo Futuro}
 Dentro de las principales tareas pendientes se encuentran:





\bibliographystyle{Thesis}
% \bibliography{bibfile}
\bibliography{bibliografia}

%\appendix


\end{document}
